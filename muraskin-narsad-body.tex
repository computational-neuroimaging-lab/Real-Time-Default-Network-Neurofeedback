\section{Introduction}

The default network (DN) consists of a constellation of brain regions implicated in spontaneous
cognition and affective regulation \cite{Raichle_2015}. DN dysregulation has been associated with a broad array of psychiatric disorders and their related symptomatology (e.g., rumination in depression \cite{Sheline2009}, attention lapses in ADHD). The lack of a specific association with any one disorder prevents disorder-centric perspectives from achieving a more comprehensive understanding of DN dysregulation as a pathophysiologic process. **Consistent with the principles of RDoC, the overarching goal of this proposal is to characterize DN dysregulation using a dimensional system that is agnostic about current disorder categories and integrates different units of analysis \cite{Insel2010}**. Central to this project is the use of a real-time fMRI (RT-fMRI) based neurofeedback experiment to measure participant's ability to modulate DN regulation. In this experiment, fMRI data is analyzed as it is being acquired to denoise the signal and subsequently decode the level of DN activity (fig. 1). The resulting measure is presented to the participant using an analog meter display and the participant is asked to use the feedback to upregulate and down-regulate their DN activity. A participant's performance is measured by how well they modulate DN activity up or down when asked. Based on associations of DN activation with mind wandering and deactivation with focused attention, we use the words ``wander'' and ``focus'' in place of up-regulate and down-regulate. RT-fMRI based indices of ability to modulate the DN, combined with task fMRI (T-fMRI) -derived profiles of DN tendency to activate and deactiva e, provide multiple units of analysis for characterizing DN function. These measures will be combined to catalogue variations in psychiatrically relevant phenotypic variables, as well as in the brain's architecture, as assessed by resting state fMRI (R-fMRI). This project entails a multi-faceted imaging study (RT-fMRI, R-fMRI, T-fMRI) in a community-ascertained sample of 60 adults (ages: 18-53 years old)\cite{Nooner2012}. A comprehensive phenotyping protocol is used to probe a range of psychiatric and cognitive domains. Consistent with the RDoC agenda, we include individuals with a range of clinical and sub-clinical psychiatric symptoms.

\section{Methods}

\subsection{Participants}
Sixty healthy volunteers with normal or corrected-to-normal vision (14 female, age: mean $\pm$ s.d. = 40.05 $\pm$ 10.82 (18-53) years) participated in the experiment. The study was conducted according to the Declaration of Helsinki and approved by the local ethics committee (Institutional Review Board of **). 15 of the subjects were collected at **Virgina Tech University** and the remaining 45 subjects were collected at the **Nathan Kline Institute**.

\subsection{Inclusion Criteria}
Minimally restrictive psychiatric exclusion criteria, which only screen out severe illness, were employed to include individuals with a range of clinical and subclinical psychiatric symptoms. Medical exclusions include: chronic medical illness, history of neoplasia requiring intrathecal chemotherapy or focal cranial irradiation, premature birth (prior to 32 weeks estimated gestational age or birth weight \textless 1500g, when available), history of neonatal intensive care unit treatment exceeding 48 hours, history of leukomalacia or static encephalopathy, or other serious neurological (specific or focal) or metabolic disorders including epilepsy (except for resolved febrile seizures), history of traumatic brain injury, stroke, aneurysm, HIV, carotid artery stenosis, encephalitis, dementia or mild cognitive impairment, Huntington's Disease, Parkinson's, hospitalization within the past month, contraindication for MRI scanning (metal implants, pacemakers, claustrophobia, metal foreign bodies or pregnancy) or inability to ambulate independently. Severe psychiatric illness can compromise the ability of an individual to comply with instructions, tolerate the MRI environment and participate in the extensive phenotyping protocol. Accordingly, participants with severe psychiatric illness were excluded, as determined by Global Assessment of Function **(GAF; DSM-IV TR (First et al., 2002))**\textless 50, history of chronic or acute substance dependence disorder,history of diagnosis with schizophrenia, history of psychiatric hospitalization, or suicide attempts requiring medical intervention.

\subsection{Phenotyping Protocol}
Participants completed a variety of assessments that probe cognitive, emotional, and behavioral domains that have been previously implicated with DMN function \cite{Andrews_Hanna2014,Buckner2008,Hamilton2011,Sheline2009}. These included the Affect Intensity Measure (AIM) (Larsen et al., 1986), Emotion Regulation Questionnaire (ERQ; (Gross and John, 2003)), Penn State Worry Questionnaire (PSWQ; (Meyer et al., 1990)), Perseverative Thinking Questionnaire (PTQ; (Ehring et al., 2011)), Positive and Negative Affect Schedule -- short version (PANAS-S; (Watson et al., 1988)), Ruminative Responses Scale (RRS; (Treynor et al., 2003)), and the Short Imaginal Process Inventory (SIPI; (Huba et al., 1981)). Assessments were completed using web-based forms implemented in COINs (Scott et al., 2011). 

\subsection{MRI Acquisition}
**NEED TO EXPLAIN VIRGINA TECH AND THEN NKI COLLECTION**
Data was acquired on a 3 T Siemens Magnetom TIM Trio scanner (Siemens Medical Solutions USA: Malvern PA, USA) using a 12-channel head coil. Anatomic images were acquired at $1 \times 1 \times 1$ mm$^{3}$ resolution with a 3D T1-weighted magnetization-prepared rapid acquisition gradient-echo (MPRAGE) sequence \cite{Mugler_1990} in 192 sagittal partitions each with a 256 $\times$ 256 field of view (FOV), 2600 ms repetition time (TR), 3.02 ms echo time (TE), 900 ms inversion time (TI), 8\textdegree flip angle (FA), and generalized auto-calibrating partially parallel acquisition (GRAPPA) \cite{Griswold_2002} acceleration factor of 2 with 32 reference lines. The sMRI data were acquired immediately after a fast localizer scan and preceded the collection of the functional data. 

A gradient echo field map sequence was acquired with the following parameters: TR 500 ms, TE1/TE2 2.72 ms/5.18 ms, FA 55\textdegree, 64 $\times$ 64 matrix, with a 220 mm FOV, 30 3.6mm thick interleaved, oblique slices, and in plane resolution of 3.4 $\times$ 3.4 mm$^{2}$. All functional data were collected with a blood oxygenation level dependent (BOLD) contrast weighted gradient-recalled echo-planar-imaging sequence (EPI) that was modified to export images, as they were acquired, to AFNI over a network interface \cite{Cox_1995,Cox_1996,LaConte_2007}. FMRI acquisition consisted of 30 3.6mm thick interleaved, oblique slices with a 10\% slice gap, TR 2000 ms, TE 30 ms, FA 90\textdegree, 64 $\times$ 64 matrix, with 220 mm FOV, and in plane resolution of 3.4 $\times$ 3.4 mm. Functional MRI scanning included a three volume ``mask'' scan and a six-minute resting state scan followed by two task scans (described later) whose order was counterbalanced across subjects.

During all scanning, galvanic skin response (GSR), pulse and respiration waveforms were measured using MRI compatible non-invasive physiological monitoring equipment (Biopac Systems, Inc.). Rate and depth of breathing were measured with a pneumatic abdominal circumference belt. Pulse was monitored with a standard infrared pulse oximeter placed on the tip of the index finger of the non-dominant hand. Skin conductance was measured with disposable passive electrodes that were non-magnetic and non-metallic, and collected on the hand. The physiological recordings were synchronized with the imaging data using a timing signal output from the scanner. Visual stimuli were presented to the participants on a projection screen that they could see through a mirror affixed to the head coil. Audio stimuli were presented through headphones using an Avotec Silent-Scan® pneumatic system (Avotec, Inc.: Stuart FL, USA).

\subsection{MRI acquisition order and online Processing}

Several stages of online processing are necessary to enable the denoising of fMRI data and decoding DMN activity levels in real-time (\textbf{Craddock et al., 2012}). These stages include calculating transforms required to convert the DMN template from MNI space to subject space, creating white matter (WM) and cerebrospinal fluid (CSF) masks for extracting nuisance signals, and training a support vector regression (SVR) model for extracting DMN activity. The MRI session was optimized to collect the data required for these various processing steps, and to perform the processing, while minimizing delays in the experiment.
After acquiring a localizer, the scanning protocol began with the acquisition of a T1 weighted anatomical image used for calculating transforms to MNI space and white matter and CSF masks. Once the image was acquired it was transferred to a DICOM server on the real-time analysis computer (RTAC), which triggered initialization of online processing. The processing script started AFNI in real-time mode, configured it for fMRI acquisition, and began structural image processing. Structural processing included reorienting the structural image to RPI voxel order, skull-stripping using AFNI's 3dSkullStrip \cite{Cox1996}, resampling the image to isotropic 2mm voxels (to reduce computational cost of subsequent operations), segmentation into grey matter (GM), white matter (WM), and cerebrospinal fluid (CSF) using FSL's FAST \cite{Zhang2001}, and normalization into MNI space using FSL's FLIRT \cite{Jenkinson2001,Jenkinson2002}. WM and CSF probability maps were binarized using a 90\% threshold. The CSF mask was constrained to the lateral ventricles using an ROI from the AAL atlas to avoid overlap with GM.
In parallel with the structural processing a FieldMap was collected but not used in the online processing. Subsequently, a three volume ``mask'' EPI scan was acquired and transferred to the RTAC. The three images were averaged, reoriented to RPI, and used to create a mask to differentiate brain signal from background (using AFNI's 3dAutomask \cite{Cox1996}). The mean image was coregistered to the anatomical image using FSL's boundary based registration (BBR) \cite{Greve2009}. The resulting linear transform was inverted and applied to the WM and CSF masks to bring them into alignment with the fMRI data. Additionally, the transform was combined with the inverted anatomical-to-MNI transform and applied to the canonical map of the DMN (from \cite{Smith2009a}).

Next, a 6-minute resting state scan (182 volumes) was collected and used as training data to create the support vector regression (SVR) model. This procedure involved motion correction followed by a nuisance regression procedure to orthogonalize the data to six head motion parameters, mean WM, mean CSF, and global signal \cite{Fox2005,Friston1996,Lund2006}. A SVR model of the DMN was trained using a modified dual regression procedure in which a spatial regression to the canonical map of the DMN was performed to extract a time course of DMN activity. The Z-transformed DMN time course was then used as labels (independent variable), with the preprocessed resting state data as features (dependent variables), for SVR training ($C=1.0$,$ \epsilon = 0.01$) using AFNI's 3dsvm tool \cite{Laconte2005}. The result was a DMN map tailored to the individual participant based on preexisting expectations about DMN anatomy and function. After SVR training was completed, the two Neurofeedback task (412 volumes) scans were run.  The order of the Neurofeedback On versus Off task based functional scans was counterbalanced and stratified for age and sex across participants.

\subsection{Resting State Scan}

Participants were instructed to keep their eyes open during the scan and fixate on a white plus ($+$) sign centered on a black background. They were asked to let their mind wander freely and if they noticed themselves focusing on any particular train of thought, to let their mind wander away from it.

\subsection{Neurofeedback Task}

In the real-time Neurofeedback (rt-NFB) task, fMRI data were processed during collection, allowing the experimenter to provide visual feedback of brain activity over the course of the experiment \cite{Cox1995,LaConte2011,McDonald2016}. The rt-NFB task was developed to examine each individual participant's ability to either increase or decrease DMN activity in response to instructions, accompanied by real-time feedback of activity from their own DMN \textbf{(Craddock et al., 2012)}.

During the rt-NFB task, participants were shown an analog meter with ``Wander'' at one end and     ``Focus'' at the other, along with an indicator of their current performance (see Fig. 3). The fixation point was a white square positioned equally between the two poles. Participants were instructed at the beginning of each block to attempt to either focus their attention (Focus) or to let their mind wander (Wander). The task began with a 30 second control condition and then proceeded with alternating blocks with a specific sequence of durations -- 30, 90, 60, 30, 90, 60, 60, 90, 30, 60, 90, and 30 seconds. At the end of each block, the participant was asked to press a button within a two second window. Starting condition (``Focus'' vs. ``Wander'') was counterbalanced, as was the location of each descriptor (``Focus'', ``Wander''; Right vs. Left) on the analog meter was counterbalanced across participants. 

Each fMRI volume acquired during the task was transmitted from the scanner to the RTAC shortly after it was reconstructed and passed to AFNI's real-time plugin \cite{Cox1995} for online processing. The volume was realigned to the previously collected mean volume to correct for motion and to bring it into alignment with the tissue masks and DMN SVR model. Mean WM intensity, mean CSF intensity, and global mean intensity were extracted from the volume using masks calculated from the earlier online segmentation of the anatomical image. A general linear model was calculated at each voxel, using all of the data that were acquired up to the current time point, with WM, CSF, global, and motion parameters included as regressors of no interest. The recently acquired volume was extracted from the residuals of the nuisance variance regression, spatially smoothed (6mm FWHM), and then a measure of DMN activity was decoded from the volume using the SVR model trained from the resting state data. The resulting measure of DMN activity was translated to an angle, which was added to the current position of the needle on the analog meter, moving it in the direction of focus or wander based on DMN activation or deactivation, respectively. This moving average procedure was used to smooth the motion of the needle. The position of the needle was reset to the center at each change between conditions. The neurofeedback stimulus was implemented in Vision Egg \cite{Straw2008} and can be downloaded from the OpenCogLab Repository (\href{http://opencoglabrepository.github.io/experiment_RTfMRIneurofeedback.html}{http://opencoglabrepository.github.io/experiment\_RTfMRIneurofeedback.html}).

\subsection{fMRI Preprocessing}

All data were preprocessed using a development version of the Configurable Pipeline for the Analysis of Connectomes (C-PAC version 0.4.0, http://fcp-indi.github.io). C-PAC is an open-source, configurable pipeline for the automated preprocessing and analysis of fMRI data \cite{Cameron2013}. C-PAC is a software implemented in Python that integrates tools from AFNI \cite{Cox1996}, FSL \cite{Smith2004}, and ANTS \cite{Avants2008} with custom tools, using the Nipype \cite{Gorgolewski2011} pipelining library, to achieve high-throughput processing on high performance computing systems.

Anatomical processing began with skull stripping using the BEaST toolset \cite{Eskildsen2012} with a study-specific library and careful manual correction of the results. The masks and BEaST library generated through this effort are shared through the Preprocessed Connectomes Project NFB Skullstripped repository (\href{http://preprocessed-connectomes-project.org/NFB_skullstripped/}{http://preprocessed-connectomes-project.org/NFB\_skullstripped/})\cite{Puccio2016}. Skullstripped images were resampled to RPI orientation and then a non-linear transform between images and a 2mm MNI brain-only template (FSL's MNI152\_T1\_2mm_brain.nii.gz, \cite{Smith_2004}) was calculated using ANTs \cite{Avants_2008}. The skullstripped images were additionally segmented into WM, GM, and CSF using FSL's FAST tool \cite{Zhang_2001}. A WM mask was calculated by applying a 0.96 threshold to the resulting WM probability map and multiplying the result by a WM prior map that was transformed into individual space using the inverse of the linear transforms previously calculated during the ANTs procedure. A CSF mask was calculated by applying a 0.96 threshold to the resulting CSF probability map and multiplying the result by a ventricle map derived from the Harvard-Oxford atlas distributed with FSL \cite{Makris_2006}. The thresholds were chosen and the priors were used to avoid overlap with grey matter.

Functional preprocessing began with resampling the data to RPI orientation, and slice timing correction. Next, motion correction was performed using a two-stage approach in which the images were first coregistered to the mean fMRI and then a new mean was calculated and used as the target for a second coregistration (AFNI 3dvolreg \cite{Cox_1999}). A 7 degree of freedom linear transform between the mean fMRI and the structural image using FSL's implementation of boundary-based registration \cite{Greve_2009}. Nuisance variable regression (NVR) was performed on motion corrected data using a 2nd order polynomial, a 24-regressor model of motion \cite{Friston_1996}, 5 nuisance signals, identified via principal components analysis of signals obtained from white matter (CompCor, \cite{Behzadi_2007}), and mean CSF signal. WM and CSF signals were extracted using the previously described masks after transforming the fMRI data to match them in 2mm space using the inverse of the linear fMRI-sMRI transform. NVR residuals were written into MNI space at 3mm resolution and subsequently smoothed using a 6mm FWHM kernel.

\subsection{Focus versus Wander Task Analysis}

Individual level analyses of the two Neurofeedback (On/Off) scans were performed in FSL using a general linear model. The expected hemodynamic response of each task condition was derived from a boxcar model, specified from stimulus onset and duration times, convolved with a canonical hemodynamic response function. Multiple regressions were performed at each voxel with fMRI activity as the independent variable and task regressors as the dependent variables. Regression coefficients at each voxel were contrasted to derive a statistic for the difference in activation between task conditions (focus \textgreater wander). The resulting individual level task maps were entered into group-level one-sample t-tests, whose significance were assessed using multiple comparison correction via a permutation test (10,000 permutations) implemented by FSL's randomise (p \textless 0.05 FWE - Threshold Free Cluster Enhancement (TFCE) \cite{Smith_2009b,Winkler_2014,Salimi_Khorshidi_2011}). Participants with a max relative RMS displacement \textgreater 1mm in either their Feedback On or Off scan were excluded from analysis. 

Maps of DMN functional connectivity were extracted from the neurofeedback scans using a dual regression procedure \cite{Filippini_2009}. Time series from 10 commonly occurring intrinsic connectivity networks (ICNs) were extracted by spatially regressing each dataset onto templates derived from a meta-analysis of task and resting state datasets \cite{Smith_2009a}. The resulting time courses were entered into voxel-wise multiple regressions to derive connectivity map for each of the 10 ICNs. To evaluate performance on the neurofeedback task, DMN time course for each participant were correlated with an ideal task time course (obtained by using instruction onset and duration information convolved with the canonical hemodynamic response function (HRF)  i.e., focus or wander).


